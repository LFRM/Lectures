\begin{darkframes}

\section{Using Libra}
\begin{frame}{The Libra Challenge}
	Create a Libra account and send 10 libras to the person of your preference. The instructions follow.  Time: 15 minutes. 
\end{frame}

\begin{frame}{Try Libra}{Currently Available in macOS and Linux}
	Assumptions:
	\begin{itemize}
	\item You are  on a Linux (Red Hat or Debian-based) or macOS system.
	\item You have a stable connection to the internet.
	\item \texttt{git} is installed on your system.
	\item Homebrew is installed on a macOS system.
	\item \texttt{yum} or \texttt{apt-get} is installed on a Linux system.
	\end{itemize}
\end{frame}

\begin{frame}{Try Libra}{Steps to Submit a Transaction}
	In this example, we'll download the necessary Libra components and execute a transaction between two users: Alice and Bob.\\~\\
	\begin{enumerate}
    	\item Clone and build Libra Core.
    	\item Build the Libra CLI client and connect to the testnet.
    	\item Create Alice’s and Bob’s accounts.
    	\item Mint coins and add to Alice’s and Bob’s accounts.
    	\item Submit a transaction.
	\end{enumerate}
\end{frame}

\defverbatim[colored]\sleepSort{
\begin{lstlisting}[language=bash,tabsize=2]
## Clone the Libra Core Repository:
git clone https://github.com/libra/libra.git

## Checkout the Testnet Branch:
git checkout testnet

## Install Dependencies:
cd libra
./scripts/dev_setup.sh
\end{lstlisting}}

\begin{frame}{Try Libra}{Clone and build Libra Core}
	\sleepSort
\end{frame}

\defverbatim[colored]\sleepSort{
\begin{lstlisting}[language=bash, tabsize=2]
## Connect to a validator node on the testnet
./scripts/cli/start_cli_testnet.sh

## Sample output
## usage: <command> <args>

## Use the following commands:

## account | a
##   Account operations
## query | q
##   Query operations
## transfer | transferb | t | tb ... 
\end{lstlisting}}

\begin{frame}{Try Libra}{Build the Libra CLI client and connect to the testnet}
	\sleepSort
\end{frame}


\defverbatim[colored]\sleepSort{
\begin{lstlisting}[language=bash, tabsize=2]
## Step 1: Check If the CLI Client Is Running on Your System

libra% account

## Sample output
## usage: account <arg>

## Use the following args for this command:

## create | c
##  Create an account. Returns reference ID to use in other operations
## list | la
##   Print all accounts that were created or loaded ... 
\end{lstlisting}}

\begin{frame}{Try Libra}{Create Alice’s and Bob’s Account}
	\sleepSort
\end{frame}

\defverbatim[colored]\sleepSort{
\begin{lstlisting}[language=bash, tabsize=2]
## Step 2: Create Alice's Account
libra% account create

## Sample output:
## >> Creating/retrieving next account from wallet
## Created/retrieved account #0 address 3ed8e5fafae4147b2...

## Step 3: Create Bob's Account
libra% account create

## Sample output:
## >> Creating/retrieving next account from wallet
## Created/retrieved account #1 address8337aac709a ...
\end{lstlisting}}

\begin{frame}{Try Libra}{Create Alice’s and Bob’s Account}
	\sleepSort
\end{frame}

\defverbatim[colored]\sleepSort{
\begin{lstlisting}[language=bash, tabsize=2]
## Step 1: Add 110 LBR to Alice's Account
libra% account mint 0 110

## Step 2: Add 52 LBR to Bob's Account
libra% account mint 1 52

## Step 3: Check the Balance
libra% query balance 0

## Sample output:
## Balance is: 110

\end{lstlisting}}
\begin{frame}{Try Libra}{Add Libra Coins to Alice's and Bob's Accounts}
	\sleepSort
\end{frame}

\defverbatim[colored]\sleepSort{
\begin{lstlisting}[language=bash, tabsize=2]
## Transfer 10 LBR from Alice to Bob
libra% transfer 0 1 10

## Check the Balance in Both Accounts After Transfer
libra% query balance 0

## Sample output:
## Balance is: 100

libra% query balance 1

## Sample output:
## Balance is: 62
\end{lstlisting}}
\begin{frame}{Try Libra}{Submit a Transaction}
	\sleepSort
\end{frame}   
\end{darkframes} 	