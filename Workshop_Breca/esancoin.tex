\begin{darkframes}
\AtBeginSection[x]{
	\begin{frame}<beamer>
     		\frametitle{Content of Section  \thesection}
      		\tableofcontents[currentsection]
    	\end{frame}}

\section{Building a Blockchain}%SmallChange}
\begin{frame}
	\begin{center}
	\includegraphics[width=75mm]{resources/esancoin2}
	\end{center}
\end{frame}	

\begin{frame}{A Cryptocurrency to Learn About Cryptocurrencies}
\begin{itemize}
	\item EsanCoin was created as an education project at Universidad ESAN, Graduate School of Business. 
	\item It is a fork of SmallChange (which is itself a fork of LiteCoin). The purpose of the project is to show that pretty much anyone with a computer can start a Litecoin/Bitcoin based currency. 
	\item Here we show how to install the cryptocurrency, start the mining process, perform transactions, and make queries to the blockchain. 
	\item All examples are performed within a private network of two nodes, but the size of the network can be extended  arbitrarily. 
	\end{itemize}
\end{frame}	

\begin{frame}{Compared to LiteCoin}
EsanCoin is a ``faster" version of Litecoin. This means we made the ``proof-of-work" protocol easy for educational purposes. %, Details follow:
	\begin{itemize}
	\item 30 seconds block targets
	\item 4 567 680 total coins
	\item 10 new coins created per generated block
	\item difficulty retargets every 5 min
	\item currently peers are looked up over IRC only
	\item currently no block checkpoints are in the code (easy to include)
	\end{itemize}
\end{frame}	

\begin{frame}{A Cryptocurrency to Learn About Cryptocurrencies}
\begin{rema}
We do not take any credit for the development of the code, outside the few changes made for making the code more readable/accessible. All credit goes to the creators of LiteCoin and SmallChange.
\end{rema}
\end{frame}	

\begin{frame}{The EsanCoin Challenge}
	Install EsanCoin and send 1 EsanCoin from your wallet to a randomly selected wallet in the group. The instructions follow. Time: 30 minutes. 
\end{frame}	


\begin{frame}{Ubuntu 18 - Linux Set Up}
\begin{itemize}
\item To implement EsanCoin you need to have at least two Ubuntu 18 - Linux machines running on the same network.
\item Your first connection should be to our server. %Its IP address is 188.166.116.72.  
\end{itemize} 
\end{frame}	

\defverbatim[colored]\sleepSort{
\begin{lstlisting}[language=bash,tabsize=2]
## Load Berkley repository libraries
sudo su
sudo add-apt-repository ppa:bitcoin/bitcoin -y 
sudo apt-get update

## Install libraries
sudo apt-get install git libboost-dev libboost-system-dev libboost-program-options-dev libboost-thread-dev libboost-filesystem-dev libcurl4-openssl-dev libdb4.8-dev libdb4.8++-dev libminiupnpc-dev qt5-default libz-dev build-essential make -y
\end{lstlisting}}

\begin{frame}{Required Libraries}
	\sleepSort
\end{frame}	

\defverbatim[colored]\sleepSort{
\begin{lstlisting}[language=bash,tabsize=2]
## Install openssl 1.0.2g 
cd /usr/local/src/
sudo wget https://www.openssl.org/source/openssl-1.0.2g.tar.gz--no-check-certificate
sudo tar -xf openssl-1.0.2g.tar.gz

cd openssl-1.0.2g
sudo ./config --prefix=/usr/local/ssl --openssldir=/usr/local/ssl shared zlib 
sudo make install
openssl version -a  
\end{lstlisting}}
\begin{frame}{Required Libraries}
	\sleepSort
\end{frame}

\defverbatim[colored]\sleepSort{
\begin{lstlisting}[language=bash,tabsize=2]
## Cleaning dependencies
cd   
sudo apt remove gcc-7 g++-7 -y  
sudo apt-get install g++-5 gcc-5 -y   
sudo apt-get install libssl1.0-dev -y  
sudo apt autoremove -y   
\end{lstlisting}}
\begin{frame}{Required Libraries}
	\sleepSort
\end{frame}

\defverbatim[colored]\sleepSort{
\begin{lstlisting}[language=bash,tabsize=2]
## Update c++ and gcc  dependencies
sudo update-alternatives --install /usr/bin/gcc gcc /usr/bin/gcc-5 10
sudo update-alternatives --install /usr/bin/gcc gcc /usr/bin/gcc-5 20
sudo update-alternatives --install /usr/bin/g++ g++ /usr/bin/g++-5 10
sudo update-alternatives --install /usr/bin/g++ g++ /usr/bin/g++-5 20
sudo update-alternatives --install /usr/bin/cc cc /usr/bin/gcc 30
sudo update-alternatives --set cc /usr/bin/gcc
sudo update-alternatives --install /usr/bin/c++ c++ /usr/bin/g++ 30
sudo update-alternatives --set c++ /usr/bin/g++
\end{lstlisting}}

\begin{frame}{Required Libraries}
	\sleepSort
\end{frame}

\defverbatim[colored]\sleepSort{
\begin{lstlisting}[language=bash,tabsize=2]
## Clone the repository
$ git clone https://github.com/esc000658/EsanCoin.git

## Locate the directory .../EsanCoin/src and execute:
$ make -f makefile.unix	 # creates wallet
$ ./EsanCoin	           # asks for a user and a password
\end{lstlisting}}

\begin{frame}{Cloning EsanCoin}
	\sleepSort
If everything went well, you will see a message requesting a user and password for your EsanCoin wallet.	
\end{frame}	

\defverbatim[colored]\sleepSort{
\begin{lstlisting}[language=bash,tabsize=2]
## Wallet credentials
rpcuser     = << unique_user >>
rpcpassword = << unique_password >>
addnode     = << IP_of_the_other_computer >>
\end{lstlisting}}

\begin{frame}{Setting Up Credentials}
Locate the directory .../EsanCoin Inside this directory, create and save a file called EsanCoin.conf with the following content:
	\sleepSort
	Note that ``IP of the other computer" refers to the IP address of any other computer connected to the network for transaction purposes (not your own computer). Use 188.166.116.72 to connect to our server. 
\end{frame}	

\begin{frame}
	\begin{center}
	\includegraphics[width=105mm]{resources/alicebob}
	\end{center}
\end{frame}

\defverbatim[colored]\sleepSort{
\begin{lstlisting}[language=bash,tabsize=2]
## Run the node
$ ./EsanCoin &        # starts the service
$ ./EsanCoin getinfo  # retrieves info about the node
$ ./EsanCoin stop     # spots the service in my node

## Start the mining:
$ ./EsanCoin setgenerate true 4 # starts multithread mining
$ ./EsanCoin getmininginfo      # retrieves mining info
$ ./EsanCoin getbalance         # checks the account balance
\end{lstlisting}}

\begin{frame}{Mining Coins}
Before making any transactions we require coins. To obtain them, return to .../EsanCoin/src and:
	\sleepSort
\end{frame}	

\defverbatim[colored]\sleepSort{
\begin{lstlisting}[language=bash,tabsize=2]
## First we need to create an account:
$ ./EsanCoin getnewaddress "name_of_account"
$ ./EsanCoin listaccounts

# Now we make our first transaction
$ ./EsanCoin move "" "name_of_account" "amount"
$ ./EsanCoin sendtoaddress "address_of_wallet" "amount"

## Making transactions of some account to other account
$ ./EsanCoin sendfrom "address_name" "address_of_wallet" "amount"
$ ./EsanCoin gettransaction "transaction_hash"
\end{lstlisting}}

\begin{frame}{Making Transactions}
Recall the mining process takes some time (5 mins).  Wait until you have some ``EsanCoins" before starting to make transactions:
	\sleepSort
\end{frame}	

\defverbatim[colored]\sleepSort{
\begin{lstlisting}[language=bash,tabsize=2]
$ ./EsanCoin listsinceblock
$ ./EsanCoin listtransactions
$ ./EsanCoin listunspent
\end{lstlisting}}

\begin{frame}{Queries in the BlockChain}
To see the blocks and transactions, use:
	\sleepSort
\end{frame}

\begin{frame}{Final Warning}
\begin{rema}
EsanCoin should not be used as a real cryptocurrency. All of the coin parameters are chosen arbitrarily or at most with ``fairness" towards everyone in mind. We stress that  EsanCoin is an education project that was build with that intention alone.
\end{rema}
\end{frame}	
\end{darkframes}